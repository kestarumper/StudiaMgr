%%% Template originaly created by Karol Kozioł (mail@karol-koziol.net) and modified for ShareLaTeX use

\documentclass[a4paper,11pt]{article}

\usepackage[T1]{fontenc}
\usepackage[utf8]{inputenc}
\usepackage{graphicx}
\usepackage{xcolor}

\renewcommand\familydefault{\sfdefault}
\usepackage{tgheros}
\usepackage[defaultmono]{droidmono}

\usepackage{amsmath,amssymb,amsthm,textcomp}
\usepackage{enumerate}
\usepackage{multicol}
\usepackage{tikz}

\usepackage{geometry}
\geometry{left=25mm,right=25mm,%
bindingoffset=0mm, top=20mm,bottom=20mm}

\usepackage[export]{adjustbox}
\usepackage{subcaption}
\usepackage{float}

\usepackage{amsmath}
\usepackage{amssymb}
\usepackage{amsthm}


\linespread{1.3}

\newcommand{\linia}{\rule{\linewidth}{0.5pt}}

% custom theorems if needed
\newtheoremstyle{mytheor}
    {1ex}{1ex}{\normalfont}{0pt}{\scshape}{.}{1ex}
    {{\thmname{#1 }}{\thmnumber{#2}}{\thmnote{ (#3)}}}

\theoremstyle{mytheor}
\newtheorem{defi}{Definition}

% my own titles
\makeatletter
\renewcommand{\maketitle}{
\begin{center}
\vspace{2ex}
{\huge \textsc{\@title}}
\vspace{1ex}
\\
\linia\\
\@author \hfill \@date
\vspace{4ex}
\end{center}
}
\makeatother
%%%

% custom footers and headers
\usepackage{fancyhdr}
\pagestyle{fancy}
\lhead{}
\chead{}
\rhead{}
\lfoot{Analiza Algorytmów, Ćwiczenia}
\cfoot{}
\rfoot{Page \thepage}
\renewcommand{\headrulewidth}{0pt}
\renewcommand{\footrulewidth}{0pt}
%

\graphicspath{ {../png/} }

%%%----------%%%----------%%%----------%%%----------%%%

\begin{document}

\title{Analiza Algorytmów, Ćwiczenia}

\author{Adrian Mucha, Politechnika Wrocławska, WPPT}

\date{26/04/2020}

\maketitle

\section*{Zad 25}
Niech, $\Lambda > 0$, $m \in \mathbb{N} > 0$, oraz niech $S_{\text{min}}^{(1)}, S_{\text{min}}^{(2)},\ldots,S_{\text{min}}^{(m)}$ będą niezależnymi zmiennymi losowymi o takim samym rozkładzie wykładniczym

$$ S_{\text{min}}^{(i)} \sim Exp(\Lambda) $$

Pokażemy, że zmienna losowa

$$ G_m = \sum_{i=1}^m S_{\text{min}}^{(i)} $$

ma rozkład gamma zdefiniowany gęstością

$$ g_m(x) = \Lambda\frac{(\Lambda x)^{m-1}}{\Gamma(m)} e^{\Lambda x},\; dla\;x > 0  $$

Z zadania 24 mamy co następuję: dla $X$ i $Y$, będących niezależnymi zmiennymi losowymi o funkcjach gęstości odpowiednio $f_X(x)$ oraz $f_Y(y)$, oraz dla $Z = X + Y$ mamy

$$ f_Z(z) = \int_{-\infty}^{\infty} f_X(x)f_Y(z - x)dx $$

w szczególności, gdy przyjmiemy $X = S_{\text{min}}^{(1)}$, $Y = S_{\text{min}}^{(2)}$ z uwzględnieniem funkcji gęstości $ f(x) = \Lambda e^{-\Lambda x} $ dla rozkładu wykładniczego mamy

\begin{align*}
    g_2(x) = \int_{0}^{x} f(x)f(x - t)dt = \int_0^x (\Lambda e^{-\Lambda(x-t)} \Lambda e^{-\Lambda t})dt = \\ \int_0^x \Lambda^2e^{-\Lambda x}dt = \Lambda^2xe^{-\Lambda x}
\end{align*}

podobnie, gdy dokonamy kwolucji x kolejną niezależną zmienną losową $S_{\text{min}}^{(3)}$ otrzymamy

\begin{align*}
    g_3(x) = \int_0^x f_2(t)f(x - t)dt = \int_0^x (\Lambda^2te^{-\Lambda t}\Lambda e^{-\Lambda(x-t)})dt = \\
    \int_0^x \Lambda^3te^{-\Lambda x}dt = \frac{1}{2}\Lambda^3x^2e^{-\Lambda x}
\end{align*}

Za pomocą indukcji matematycznej, udowodnimy następujący wzór

\begin{align*}
    g_m(x) = \frac{1}{(m-1)!}\Lambda^mx^{m-1}e^{-\Lambda x}
\end{align*}


\begin{proof}
    dla $m = 1$, wzór zachodzi
    
    \begin{align*}
        g_1(x) = \frac{1}{(1-1)!}\Lambda^1x^{1-1}e^{-\Lambda x} = \Lambda e^{-\Lambda x} = f(x)
    \end{align*}
    
    Załóżmy, że $g_m(x)$ jest prawdziwa dla sumy $m$ niezależnych zmiennych $G_m = \sum_{i=1}^m S_{\text{min}}^{(i)}$. Następnie, dodając kolejną niezależną zmienną losową $S_{\text{min}}^{(m+1)} \sim Exp(\Lambda)$ o funkcji gęstości $f(x)$, otrzymujemy
    
    \begin{align*}
        \int_0^x g_m(t)f(x-t)dt =
        \int_0^x \frac{1}{(m-1)!} \Lambda^m t^{m-1} e^{-\Lambda t} \Lambda e^{-\Lambda (x-t)} dt = \\
        \frac{1}{(m-1)!} \Lambda^m \Lambda e^{-\Lambda x} \int_0^x t^{m-1} dt =                    \\
        \frac{1}{m} \cdot \frac{1}{(m-1)!} \Lambda^{m+1} x^m e^{-\Lambda x} =                      \\
        \frac{1}{m!} \Lambda^{m+1} x^m e^{-\Lambda x} = g_{m+1}(x)
    \end{align*}
\end{proof}

Ostatecznie, podstawiając $\Gamma(m) = (m-1)!$, do wzoru otrzymujemy:

\begin{align*}
    g_m(x) = \frac{1}{(m-1)!}\Lambda^mx^{m-1}e^{-\Lambda x} = \Lambda\frac{(\Lambda x)^{m-1}}{\Gamma(m)}e^{-\Lambda x}
\end{align*}

\end{document}