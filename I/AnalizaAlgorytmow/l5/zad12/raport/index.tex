\documentclass[a4paper,11pt]{article}

\usepackage[T1]{fontenc}
\usepackage[utf8]{inputenc}
\usepackage{graphicx}
\usepackage{xcolor}

\renewcommand\familydefault{\sfdefault}
\usepackage{tgheros}

\usepackage{amsmath,amssymb,amsthm,textcomp}
\usepackage{enumerate}
\usepackage{multicol}
\usepackage{tikz}

\usepackage{geometry}
\geometry{left=25mm,right=25mm,%
bindingoffset=0mm, top=20mm,bottom=20mm}

\usepackage[export]{adjustbox}
\usepackage{subcaption}
\usepackage{float}


\linespread{1.3}

\newcommand{\linia}{\rule{\linewidth}{0.5pt}}

% custom theorems if needed
\newtheoremstyle{mytheor}
    {1ex}{1ex}{\normalfont}{0pt}{\scshape}{.}{1ex}
    {{\thmname{#1 }}{\thmnumber{#2}}{\thmnote{ (#3)}}}

\theoremstyle{mytheor}
\newtheorem{defi}{Definition}

% my own titles
\makeatletter
\renewcommand{\maketitle}{
\begin{center}
\vspace{2ex}
{\huge \textsc{\@title}}
\vspace{1ex}
\\
\linia\\
\@author \hfill \@date
\vspace{4ex}
\end{center}
}
\makeatother
%%%

% custom footers and headers
\usepackage{fancyhdr}
\pagestyle{fancy}
\lhead{}
\chead{}
\rhead{}
\lfoot{Analiza Algorytmów, lab 5}
\cfoot{}
\rfoot{Page \thepage}
\renewcommand{\headrulewidth}{0pt}
\renewcommand{\footrulewidth}{0pt}
%

\graphicspath{ {../data/png/} }

%%%----------%%%----------%%%----------%%%----------%%%

\begin{document}

\title{Analiza Algorytmów, lab 5}

\author{Adrian Mucha, Politechnika Wrocławska, WPPT}

\date{05/06/2020}

\maketitle

\section*{Zad 12, Mutual Exclusion}
Weryfikacja przejść z dowolnej konfiguracji do legalnej wymaga sprawdzenia każdej konfiguracji początkowej. Kolejnym problemem okazało się wygenerowanie \textit{ścieżek} reprezentujących konkretne \textit{wykonanie} (kolejność), które składane jest na podstawie wygenerowanych \textit{rund}.

Szczególną przeszkodą okazało się zarządzanie pamięcią oraz limity z tym związane. Warto zaznaczyć, że wszystkich stanów początkowych jest $(n+1)^n$. Dodatkowo, każdy taki stan można przekształcać w rundzie na $n!$ sposobów, czyli tyle ile mamy sposobów na zakolejkowanie $n$ procesów. Sposób działania przypomina przeszukiwanie drzewa wszerz.

Pomimo dołożenia wszelkich starań optymalizacyjnych, udało się jedynie znaleźć rozwiązanie dla $n = 4$, gdzie najdłuższa ścieżka była długości $15$. W przypadku $n = 5$, zużycie pamięci szybko przekroczyło \texttt{8GiB} pamięci \texttt{RAM} i uniemożliwiło dalsze obliczenia.

Usprawnienia optymalizacyjne polegały na zaznaczaniu już odwiedzonych wierzchołków oraz zapisywania w nich najdłuższej już znalezionej ścieżki do stanu stabilnego.
Oczywistym krokiem było wykluczenie rekursji gdyż szybko doprowadziłaby ona do przepełnienia stosu.


\end{document}