%%% Template originaly created by Karol Kozioł (mail@karol-koziol.net) and modified for ShareLaTeX use

\documentclass[a4paper,11pt]{article}

\usepackage[T1]{fontenc}
\usepackage[utf8]{inputenc}
\usepackage{graphicx}
\usepackage{xcolor}

\renewcommand\familydefault{\sfdefault}
\usepackage{tgheros}

\usepackage{amsmath,amssymb,amsthm,textcomp}
\usepackage{enumerate}
\usepackage{multicol}
\usepackage{tikz}

\usepackage{algpseudocode}
\usepackage{algorithm}

\usepackage{geometry}
\geometry{left=25mm,right=25mm,%
bindingoffset=0mm, top=20mm,bottom=20mm}


\linespread{1.3}

\newcommand{\linia}{\rule{\linewidth}{0.5pt}}

% custom theorems if needed
\newtheoremstyle{mytheor}
    {1ex}{1ex}{\normalfont}{0pt}{\scshape}{.}{1ex}
    {{\thmname{#1 }}{\thmnumber{#2}}{\thmnote{ (#3)}}}

\theoremstyle{mytheor}
\newtheorem{defi}{Definition}

% my own titles
\makeatletter
\renewcommand{\maketitle}{
\begin{center}
\vspace{2ex}
{\huge \textsc{\@title}}
\vspace{1ex}
\\
\linia\\
\@author \hfill \@date
\vspace{4ex}
\end{center}
}
\makeatother
%%%

% custom footers and headers
\usepackage{fancyhdr}
\pagestyle{fancy}
\lhead{}
\chead{}
\rhead{}
\cfoot{}
\rfoot{Page \thepage}
\renewcommand{\headrulewidth}{0pt}
\renewcommand{\footrulewidth}{0pt}
%

% code listing settings
\usepackage{listings}
\lstset{
    language=Python,
    basicstyle=\ttfamily\small,
    aboveskip={1.0\baselineskip},
    belowskip={1.0\baselineskip},
    columns=fixed,
    extendedchars=true,
    breaklines=true,
    tabsize=4,
    prebreak=\raisebox{0ex}[0ex][0ex]{\ensuremath{\hookleftarrow}},
    frame=lines,
    showtabs=false,
    showspaces=false,
    showstringspaces=false,
    keywordstyle=\color[rgb]{0.627,0.126,0.941},
    commentstyle=\color[rgb]{0.133,0.545,0.133},
    stringstyle=\color[rgb]{01,0,0},
    numbers=left,
    numberstyle=\small,
    stepnumber=1,
    numbersep=10pt,
    captionpos=t,
    escapeinside={\%*}{*)}
}

%%%----------%%%----------%%%----------%%%----------%%%

\begin{document}

\title{Geometria obliczeniowa - Ćwiczenia}

\author{Adrian Mucha 236526, Politechnika Wrocławska WPPT}

\date{19.05.2020}

\maketitle

\section*{Zadanie 11}
Podaj algorytm liniowy obliczający 3-kolorowanie striangulowanego wielokąta prostego.
\subsubsection*{Rozwiązanie}
Niech $T_P$ będzie traingulacją wielokąta $P$. Wybierzemy podzbiór wierzchołków $P$ taki, że każdy trójkąt w $T_P$ ma co najmniej jeden wybrany wierzchołek i umieścimy w tych wierzchołkach kamery. Każdemu wierzchołkowi nadamy jeden z kolorów \texttt{czarny}, \texttt{szary} lub \texttt{biały} w taki sposób by dowolne dwa wierzchołki połączone krawędzią lub przekątną miały różne kolory. Każdy trójkąt wielokąta ma wierzchołek \texttt{czarny}, \texttt{szary} oraz \texttt{biały}. Kamery umieścimy we wszystkich wierzchołkach tego samego koloru, który jest najmniej liczny dzięki czemu możemy chronić $P$, używając co najwyżej $\lfloor\frac{n}{3}\rfloor$ kamer.

Aby dokonać 3-kolorowania wielokąta, wykorzystamy graf dualny $T_P$, oznaczmy go jako $G(T_P)$. Wierzchołkami grafu będą trójkąty w $T_P$. Trójkąt odpowiadający węzłowi $v$ oznaczamy jako $t(v)$. Krawędź między węzłami $v$ i $\mu$ istnieje gdy $t(v)$ i $t(\mu)$ rozdziela przekątna. Krawędź w $G(T_P)$ odpowiada przekątnej w $T_P$. $G(T_P)$ jest drzewem. Kolorowanie grafu możemy zatem znaleźć za pomocą przeszukiwania w głąb. Podczas przeszukiwania w głąb zachowujemy następujący niezmiennik: \textbf{wszystkie wierzchołki napotkanych dotąd trójkątów zostały pomalowane na biało, szaro, lub czarno i żadne dwa połączone wierzchołki nie otrzymały tego samego koloru}. Przechodząc z wierzchołka $\mu$ do wierzchołka $v$, $t(v)$ oraz $t(\mu)$ rozdziela przekątna. Ponieważ wierzchołki $t(\mu)$ zostały już pomalowane, więc tylko jeden wierzchołek $t(v)$ został do pokolorowania (można to zrobić na jeden sposób). \\

\begin{algorithm}
    \begin{algorithmic}[1]
        \Function{VisitNode}{\textit{Wierzchołek} $\mu$}
        % \State wybierz dowolny wierzchołek startowy $\mu \in G(T_P)$
        \State oznacz $\mu$ jako odwiedzony
        \For{dla każdego wierzchołka $v \in$ \texttt{neighbours}($\mu$)}
        \If{$v$ jest nieodwiedzony}
        \State pokoloruj przeciwległy wierzchołek na pozostały kolor niekolidujący z wierzchołkami na krawędzi dzielącej $t(v)$ oraz $t(\mu)$
        \State VisitNode($v$)
        \EndIf
        \EndFor
        \EndFunction
        
        \Function{DepthFirstSearch}{\textit{Graf} $G$}
        \For{dla każdego wierzchołka $\mu \in G$}
        \State oznacz $\mu$ jako nieodwiedzony
        \EndFor
        \State wybierz dowolny $\mu \in G$
        \State pokoloruj wierzchołki trójkąta $t(\mu)$
        \State VisitNode($\mu$)
        \EndFunction \\\\
        \Call{DepthFirstSearch}{$G(T_P)$}\\
        \Return ilość wierzchołków koloru najmniej licznego
    \end{algorithmic}
    \caption{3-kolorowanie striangulowanego wielokąta}
    Procedura \texttt{DepthFirstSearch} ma złożoność $\mathcal{O}(|V|)$ gdyż wraz z podprocedurą $\texttt{VisitNode}$ odwiedzają każdy wierzchołek jeden raz. Obliczenie \textit{trzeciego brakującego koloru} odbywa się w czasie $\mathcal{O}(1)$.
\end{algorithm}

\end{document}