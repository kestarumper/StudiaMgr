%%% Template originaly created by Karol Kozioł (mail@karol-koziol.net) and modified for ShareLaTeX use

\documentclass[a4paper,11pt]{article}

\usepackage[T1]{fontenc}
\usepackage[utf8]{inputenc}
\usepackage{graphicx}
\usepackage{xcolor}

\renewcommand\familydefault{\sfdefault}
\usepackage{tgheros}

\usepackage{amsmath,amssymb,amsthm,textcomp}
\usepackage{enumerate}
\usepackage{multicol}
\usepackage{tikz}

\usepackage{geometry}
\geometry{left=25mm,right=25mm,%
bindingoffset=0mm, top=20mm,bottom=20mm}

\usepackage[export]{adjustbox}
\usepackage{subcaption}
\usepackage{float}

\usepackage{amsmath}
\usepackage{amssymb}
\usepackage{amsthm}

\usepackage{geometry}
\geometry{left=25mm,right=25mm,%
bindingoffset=0mm, top=20mm,bottom=20mm}


\linespread{1.3}

\newcommand{\linia}{\rule{\linewidth}{0.5pt}}

% custom theorems if needed
\newtheoremstyle{mytheor}
    {1ex}{1ex}{\normalfont}{0pt}{\scshape}{.}{1ex}
    {{\thmname{#1 }}{\thmnumber{#2}}{\thmnote{ (#3)}}}

\theoremstyle{mytheor}
\newtheorem{defi}{Definition}

% my own titles
\makeatletter
\renewcommand{\maketitle}{
\begin{center}
\vspace{2ex}
{\huge \textsc{\@title}}
\vspace{1ex}
\\
\linia\\
\@author \hfill \@date
\vspace{4ex}
\end{center}
}
\makeatother
%%%

% custom footers and headers
\usepackage{fancyhdr}
\pagestyle{fancy}
\lhead{}
\chead{}
\rhead{}
\cfoot{}
\rfoot{Page \thepage}
\renewcommand{\headrulewidth}{0pt}
\renewcommand{\footrulewidth}{0pt}
%

%%%----------%%%----------%%%----------%%%----------%%%

\begin{document}

\title{Geometria obliczeniowa - Ćwiczenia}

\author{Adrian Mucha 236526, Politechnika Wrocławska WPPT}

\date{06.06.2020}

\maketitle

\section*{Zadanie 15}
Załóżmy, że kolejne $n$ wierzchołków wielokąta wypukłego zostały podane w tablicy. Pokazać, że w czasie $O(\log n)$ można sprawdzić, czy dany punkt leży wewnątrz tego wielokąta.

\subsection*{Rozwiązanie}
Ustalmy punkt z najmniejszą składową $x$. Jeżeli istnieje kilka, wybieramy tę z najmniejszą składową $y$. Oznaczmy ten punkt jako $p_0$. Reszta punktów to $p_1,\ldots, p_n$ w kolejności według ich kąta od ustalonego punktu (wielokąt jest numerowany przeciwnie do ruchu wskazkówek zegara).

Jeżeli punkt należy do wielokąto, to należy do jakiegoś trójkąta $p_0, p_i, p_{i+1}$ (lub dwóch, jeśli leży na brzegu trójątów). Rozważmy trójkąt $p_0, p_i, p_{i+1}$, taki że $p$ należy do tego trójkąta, oraz $i$ jest największe spośród takich trójkątów.

Istnieje specjalny przypadek, że $p$ leży na odcinku $(p_0, p_n)$, ale będzie on rozważany osobno. W pozostałych przypadkach, wszystkie punkty $p_j$ dla których $j\leq i$ są po lewej stronie $p$ (przeciwnie do ruchu wskazkówek zegara) w stosunku do $p_0$ oraz wszystkie pozostałe, które nie leżą przeciwnie do ruchu wskazówek w stosunku do $p$. Dzięki tej własności można stosować \textit{przeszukiwanie binarne} by znaleźć punkt $p_i$, taki że $p_i$ nie leży przeciwnie do ruchu wskazówek zegara od $p$ w stosunku do $p_0$ oraz $i$ jest największe spośród tych punktów. Na sam koniec sprawdzane jest czy rzeczywiście punkt $p$ leży wewnątrz znalezionego trójkąta.

Znak $(a-c) \times (b-c)$ mówi o tym, czy punkt $a$ jest po lewej (przeciw wskazówkom zegara) lub po prawej (zgodnie ze wskazówkami zegara) od punktu $b$ w stosunku do punktu $c$. 

\begin{enumerate}
    \item $(a-c) \times (b-c) > 0$ - punkt $a$ leży na prawo od wektora biegnącego z $c$ do $b$, co oznacza ruch zgodny ze wskazówkami zegara od $b$ w stosunku do $c$.
    \item $(a-c) \times (b-c) < 0$ - punkt $a$ leży na lewo od wektora biegnącego z $c$ do $b$, co oznacza ruch przeiwny do wskazówkek zegara
\end{enumerate}

Rozważmy zapytanie o punkt $p$. Najpierw sprawdzamy czy punkt leży między $p_1$ a $p_n$. W przeciwnym wypadku, będziemy wiedzieć że nie może być częścią wielokąta. Można to sprawdzić sprawdzając iloczyn wektorowy $(p_1 - p_0) \times (p - p_0) = 0$ lub ma ten sam znak co $(p_1 - p_0) \times (p_n - p_0)$ oraz że $(p_n - p_0) \times (p - p_0) = 0$ lub ma ten sam znak co $(p_n - p_0) \times (p_1 - p_0)$. Następnie sprawdzamy specjalny przypadek gdy $p$ leży na $(p_0, p_1)$. Ostatecznie szukamy ostatniego punktu z $p_1,\ldots,p_n$, który nie jest przeciwny do ruchu wskazówek zegara od $p$ w stosunku do $p_0$. Dla punktu $p_i$ sprawdzamy to za pomocą $(p_i - p_0) \times (p i p_0) \leq 0$. Po znalezieniu takiego punktu $p_i$, sprawdzamy czy $p$ leży wewnątrz trójkąta $p_0, p_i, p_{i+1}$.

\end{document}