\section*{Zad 1}
W zadaniu użyto własny liniowy generator kongruencyjny.

$$ lcg_{k+1} = (m \cdot lcg_{k} + c) \mod n = 15623 \cdot lcg_n + 1836716 \mod 21373737$$

Distinguisher \emph{oracle} potrafi przewidywać wartości generowane przez $lcg$ na podstawie zaobserwowanych wcześniej liczb. Distinguisher potrafi obliczyć składowe \emph{c}, \emph{m} oraz \emph{n}.

\subsubsection*{Wszystkie znane komponenty}
Wystarczy obliczyć kolejny stan generatora - przypadek trywialny.
\subsubsection*{Składowa $c$ nieznana}
Znając $2$ stany $s_k, s_{k+1}$ możemy obliczyć $c$.
\begin{align*}
    s_{k+1} = s_k m + c \mod n \\
    c = s_{k+1} - s_k m \mod n
\end{align*}

\subsubsection*{Składowa $m$ nieznana}
Znając $3$ stany $s_k, s_{k+1}, s_{k+2}$ możemy obliczyć $m$
\begin{align*}
    s_1 = s_0 m + c  \mod n \\
    s_2 = s_1 m + c  \mod n \\
    s_2 - s_1 = s_1 m - s_0*m  \mod n \\
    s_2 - s_1 = m(s_1 - s_0)  \mod n \\
    m = (s_2 - s_1)(s_1 - s_0)^{-1} \mod n \\
\end{align*}
Jedyną trudnością jest tutaj obliczanie odwrotności modulo, którą otrzymuje się poprzez zastosowanie rozszerzonego algorytmu Euklidesa.
\begin{align*}
    ax + my = gcd(a,m) = 1 \\
    ax - 1 = (-y)m \\
    ax \equiv 1 \mod m
\end{align*}

\subsubsection*{Składowa $n$ nieznana}
Znając $6$ stanów $s_k, s_{k+1},\ldots s_{k+5}$ możemy obliczyć $n$.

Skorzystamy z dwóch własności
\begin{align}
    X = 0 \mod n \\
    X = kn \mod n
\end{align}
oraz faktu, że z dużym prawdopodobieństwem, jeśli mamy kilka losowych liczb będących wielokrotnościami $n$, to ich największym wspólnym dzielnikiem będzię $n$.

Następnie możemy wprowadzić pewną sekwencję $T(n) = S(n+1) - S(n)$
\begin{align*}
    t_0 = s_1 - s_0 \\
    t_1 = s_2 - s_1 = (s_1m+c) - (s_0m+c) = m(s_1 - s_0) = mt_0 \mod n \\
    t_2 = s_3 - s_2 = (s_2m+c) - (s_1m+c) = m(s_2 - s_1) = mt_1 \mod n \\
    t_3 = s_4 - s_3 = (s_3m+c) - (s_2m+c) = m(s_3 - s_2) = mt_2 \mod n \\
\end{align*}
oraz wykorzystać wcześniej wspomniany fakt by obliczyć

$$ t_2 t_0 - t_1 t_1 = (mmt_0 \cdot t_0) - (mt_0 \cdot mt_0) = 0 \mod n $$

a następnie obliczamy $gcd$ i otrzymujemy moduł.

Na koniec warto wspomnieć, że program sprowadza przypadki od najtrudniejszego (znalezienie modułu), przez znalezienie mnożnika i na końcu do znalezienia stałej. W przypadku gdy tylko jedna z nich jest nieznana to reszta nie jest obliczana.