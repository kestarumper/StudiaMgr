\section*{Zad 2}

\subsection*{Opis glibc random}
Na początek przybliżymy metodę działania \texttt{glibc}. Dla zadanego ziarna $s$, wektor inicjalizujący $r_0,r_1,\ldots,r_33$ jest obliczany według następującego schematu:

\begin{itemize}
    \item $r_0 = s$
    \item $\forall_{i\in \{1,\ldots,30\}}\hspace{8pt} r_i = (16807 \cdot r_{i-1}) \mod (2^{31} - 1)$
    \item $\forall_{i\in \{31,\ldots,33\}}\hspace{4pt} r_i = r_{i - 31}$
\end{itemize}

Następnie, sekwencja pseudolosowa $r_{34},r_{35},\ldots$ jest generowana dzięki pętli liniowego sprzężenia zwrotnego

\begin{itemize}
    \item $\forall_{i\geq 34}\hspace{8pt} r_i = (r_{i-3} + r_{i-31}) \mod 2^{32}$
\end{itemize}

$r_0,\ldots,r_{343}$ dezintegrujemy, a $i$-tym wyjściem $o_i$ funkcji \texttt{myrand()} jest

\begin{itemize}
    \item $o_i = r_{i+344} >> 1$
\end{itemize}

$>>$ jest tutaj przesunięciem bitowym w prawo (pozbywamy się najmniejznaczącego bitu).

\subsection*{Liniowość}
Pomimo odrzucanego ostatniego bitu, mamy prawie liniową zależność otrzymywanej sekwencji wychodzącej. Mamy więc $\forall_{i\geq 31}$
$$ o_i = o_{i-31} + o_{i-3} \mod 2^{31} - 1 $$
lub
$$ o_i = o_{i-31} + o_{i-3} + 1 \mod 2^{31} - 1 $$
Nasz \textit{distinguisher} potrafi więc odgadnąć wynik funkcji \texttt{glibc random}.