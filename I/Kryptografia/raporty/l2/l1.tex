%%% Template originaly created by Karol Kozioł (mail@karol-koziol.net) and modified for ShareLaTeX use

\documentclass[a4paper,11pt]{article}

\usepackage[T1]{fontenc}
\usepackage[utf8]{inputenc}
\usepackage{graphicx}
\usepackage{xcolor}

\renewcommand\familydefault{\sfdefault}
\usepackage{tgheros}

\usepackage{amsmath,amssymb,amsthm,textcomp}
\usepackage{enumerate}
\usepackage{multicol}
\usepackage{tikz}

\usepackage{geometry}
\geometry{left=25mm,right=25mm,%
bindingoffset=0mm, top=20mm,bottom=20mm}

\usepackage[export]{adjustbox}
\usepackage{subcaption}
\usepackage{float}

\usepackage{amsmath}
\usepackage{amssymb}
\usepackage{amsthm}


\linespread{1.3}

\newcommand{\linia}{\rule{\linewidth}{0.5pt}}

% custom theorems if needed
\newtheoremstyle{mytheor}
 {1ex}{1ex}{\normalfont}{0pt}{\scshape}{.}{1ex}
 {{\thmname{#1 }}{\thmnumber{#2}}{\thmnote{ (#3)}}}

\theoremstyle{mytheor}
\newtheorem{defi}{Definition}

% my own titles
\makeatletter
\renewcommand{\maketitle}{
\begin{center}
\vspace{2ex}
{\huge \textsc{\@title}}
\vspace{1ex}
\\
\linia\\
\@author \hfill \@date
\vspace{4ex}
\end{center}
}
\makeatother
%%%

% custom footers and headers
\usepackage{fancyhdr}
\pagestyle{fancy}
\lhead{}
\chead{}
\rhead{}
\lfoot{Kryptografia, Lista 2}
\cfoot{}
\rfoot{Page \thepage}
\renewcommand{\headrulewidth}{0pt}
\renewcommand{\footrulewidth}{0pt}
%

\graphicspath{ {../png/} }

%%%----------%%%----------%%%----------%%%----------%%%

\begin{document}

\title{Kryptografia, Lista 2}

\author{Adrian Mucha, Politechnika Wrocławska, WPPT}

\date{17/05/2020}

\maketitle

\section*{AES vs. DES}
% Please add the following required packages to your document preamble:
% \usepackage{graphicx}
\begin{table}[H]
    \begin{tabular}{|p{0.5\textwidth}|p{0.5\textwidth}|}
        \hline
        \multicolumn{1}{|c|}{\textbf{AES}}                                                 & \multicolumn{1}{c|}{\textbf{DES}}                                                                         \\ \hline
        AES: Advanced Encryption Standard                                                  & DES: Data Encryption Standard                                                                             \\ \hline
        Klucze mają długość 128, 192 lub 256 bitów                                         & Długość klucza to 56 bitów.                                                                               \\ \hline
        Liczba rund w zależności od klucza: 10(128-bitów), 12(192-bitów) lub 14(256-bitów) & 16 rund identycznych operacji                                                                             \\ \hline
        Struktura opiera się o sieć substytucyjno-permutacyjną.                            & Struktura opiera się o sieci Feistela.                                                                    \\ \hline
        AES jest bezpieczniejszy niż DES i jest światowym standardem.                      & DES może być łatwo złamany i ma wiele znanych słabości. Istnieje 3DES który jest bezpieczniejszy niż DES. \\ \hline
        Koduje 128 bitów tekstu jawnego.                                                   & Koduje 64 bity tekstu jawnego.                                                                            \\ \hline
        Brak znanych ataków crypto-analitycznych prócz ataków side channel.                & Znane ataki: Brute-force, Linear crypt-analysis oraz Differential crypt-analysis.                         \\ \hline
    \end{tabular}
    \label{tab:my-table}
\end{table}

\section*{Tryby AES}
AES obsługuje różne tryby operowania na danych, które posiadają różne właściwości i stopnie bezpieczeństwa oraz szybkości działania czy możliwości pracy równoległej na wielu blokach.

\begin{table}[H]
    \begin{tabular}{|p{0.1\textwidth}|p{0.45\textwidth}|p{0.45\textwidth}|}
        \hline
        \multicolumn{1}{|c}{\textbf{Tryb}} & \multicolumn{1}{|c|}{\textbf{Zalety}} & \multicolumn{1}{|c|}{\textbf{Wady}}
        \\ \hline
        ECB & \begin{itemize}
            \setlength\itemsep{-0.5em}
            \item Prosty
            \item Szybki
            \item Równoległy
        \end{itemize} & \begin{itemize}
            \setlength\itemsep{-0.8em}
            \item Powtórzenia tekstu jawnego będą widoczne w szyfrze
            \item Uszkodzony szyfr będzie mieć wpływ na tekst jawny
            \item Brak odporności na \textit{replay attacks}
            \item Nie powinno się go używać
        \end{itemize}
        \\ \hline

        CBC & \begin{itemize}
            \setlength\itemsep{-0.5em}
            \item Równoległe odszyfrowywanie
            \item Powtórzenia nie będą widoczne w szyfrze
        \end{itemize} & \begin{itemize}
            \setlength\itemsep{-0.8em}
            \item Brak równoległego szyfrowania
            \item Uszkodzony blok wpływa na kolejne bloki
        \end{itemize}
        \\ \hline

        CFB & \begin{itemize}
            \setlength\itemsep{-0.5em}
            \item Brak wyrównania (no padding)
            \item Równoległe odszyfrowywanie
        \end{itemize} & \begin{itemize}
            \setlength\itemsep{-0.8em}
            \item Brak równoległego szyfrowania
            \item Brak odporności na \textit{replay attack}
            \item Uszkodzony blok wpływa na kolejne bloki
        \end{itemize}
        \\ \hline

        OFB & \begin{itemize}
            \setlength\itemsep{-0.5em}
            \item Brak wyrównania (no padding)
            \item Szyfrowanie i deszyfrowanie używa tego samego schematu
            \item Uszkodzony blok nie wpływa na inne
        \end{itemize} & \begin{itemize}
            \setlength\itemsep{-0.8em}
            \item Brak równoległego szyfrowania
            \item Adwersarz może zmienić uszkodzić część szyfru by zmienić tekst jawny
        \end{itemize}
        \\ \hline
        
        CTR & \begin{itemize}
            \setlength\itemsep{-0.5em}
            \item Brak wyrównania (no padding)
            \item Równoległe szyfrowanie i deszyfrowanie
            \item Szyfrowanie i deszyfrowanie używa tego samego schematu
            \item Uszkodzony blok nie wpływa na inne
        \end{itemize} & \begin{itemize}
            \setlength\itemsep{-0.8em}
            \item Adwersarz może zmienić uszkodzić część szyfru by zmienić tekst jawny
        \end{itemize}
        \\ \hline
    \end{tabular}
    \label{tab:my-table}
\end{table}

\end{document}