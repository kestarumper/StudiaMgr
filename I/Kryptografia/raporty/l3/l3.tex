%%% Template originaly created by Karol Kozioł (mail@karol-koziol.net) and modified for ShareLaTeX use

\documentclass[a4paper,11pt]{article}

\usepackage[T1]{fontenc}
\usepackage[utf8]{inputenc}
\usepackage{graphicx}
\usepackage{xcolor}

\renewcommand\familydefault{\sfdefault}
\usepackage{tgheros}

\usepackage{amsmath,amssymb,amsthm,textcomp}
\usepackage{enumerate}
\usepackage{multicol}
\usepackage{tikz}

\usepackage{geometry}
\geometry{left=25mm,right=25mm,%
bindingoffset=0mm, top=20mm,bottom=20mm}

\usepackage[export]{adjustbox}
\usepackage{subcaption}
\usepackage{float}

\usepackage{amsmath}
\usepackage{amssymb}
\usepackage{amsthm}


\linespread{1.3}

\newcommand{\linia}{\rule{\linewidth}{0.5pt}}

% custom theorems if needed
\newtheoremstyle{mytheor}
 {1ex}{1ex}{\normalfont}{0pt}{\scshape}{.}{1ex}
 {{\thmname{#1 }}{\thmnumber{#2}}{\thmnote{ (#3)}}}

\theoremstyle{mytheor}
\newtheorem{defi}{Definition}

% my own titles
\makeatletter
\renewcommand{\maketitle}{
\begin{center}
\vspace{2ex}
{\huge \textsc{\@title}}
\vspace{1ex}
\\
\linia\\
\@author \hfill \@date
\vspace{4ex}
\end{center}
}
\makeatother
%%%

% custom footers and headers
\usepackage{fancyhdr}
\pagestyle{fancy}
\lhead{}
\chead{}
\rhead{}
\lfoot{Kryptografia, Lista 3}
\cfoot{}
\rfoot{Page \thepage}
\renewcommand{\headrulewidth}{0pt}
\renewcommand{\footrulewidth}{0pt}
%

%%%----------%%%----------%%%----------%%%----------%%%

\begin{document}

\title{Kryptografia, Lista 3}

\author{Adrian Mucha, Politechnika Wrocławska, WPPT}

\date{26.05.2020}

\maketitle

\section*{Merkle Hellman Kryptosystem}
\subsection*{Wstęp}
Rozważmy 0-1 problem plecakowy z $n$ przedmiotami o wartościach $v_i$ oraz wagach $w_i$. Mamy znaleźć podzbiór rzeczy maksymalizujący zysk nieprzekraczając pewnej maksymalnej pojemności $W$. Problem jest NP-trudny, ale może zostać rozwiązany w czasie pseudo-wielomianowym z zastosowaniem programowania dynamicznego.

Problem sumy podzbioru (\textit{subset sum problem}) jest specjalnym przypadkiem problemu plecakowego gdzie każda wartość jest równa swej wadze. Wejściem jest zbiór $A = \{a_1,\ldots,a_n\ | a_i\in \mathbb{N}_{>0} \}$ oraz liczba dodatnia $S$. Jeżeli istnieje podzbiór $A$ sumujący się do $S$ to wyjściem jest \texttt{TRUE}, \texttt{FALSE} w p.p. Ten problem również jest NP-trudny.

Łatwy problem plecakowy to taki, w którym zbiór $A$ jest ciągiem super-rosnącym, tj $$ a_2 > a_1, a_3 > a_2 + a_1,\ldots, a_n > a_{n-1},\ldots, a_1 $$. Przykładem takiego ciągu jest ciąg potęgowy $2^n$. Z tego zbioru wybieramy podzbiór $X\subset A$ sumujący się do $E$, który będziemy traktować jako klucz prywatny.

\subsection*{Komunikacja}
Alicja:
\begin{enumerate}
    \item Generuje sekretny \textit{klucz prywatny}
    \item Generuje \textit{klucz publiczny}, który jest dostępny dla wszystkich
    \item Otrzymuje zaszyfrowaną wiadomość od Boba
    \item Odszyfrowuje ją za pomocą \textit{klucza prywatnego}
\end{enumerate}
Bob:
\begin{enumerate}
    \item Używa \textit{klucza publicznego} Alicji do zaszyfrowania tekstu jawnego
    \item Wysyła zaszyfrowany tekst do Alicji
\end{enumerate}

\subsection*{Algorytm}
Alicja:
\begin{enumerate}
    \item Wybiera $A = \{a_1, a_2,\ldots,a_n\ | a_i > \sum_{1 \leq j < i}a_j\}$ (super-rosnący ciąg), wybiera prosty problem plecakowy
    \item Oblicza $E = \sum_{i = 1}^n a_i$
    \item Wybiera $M > E$
    \item Wybiera $W$, takie że $2 \leq W < M$ oraz $gcd(W,M) = 1$ by zapewnić odwracalność modulo $M$.
    \item Oblicza publiczny (trudny) problem plecakowy $B = \{b_1, b_2,\ldots,b_n\}$, w którym $$b_i = Wa_i \mod M $$
    \item Zachowuje (ukrywa) \textit{klucz prywatny} $(A, W, M)$
    \item Publikuje \textit{klucz publiczny} $B$
\end{enumerate}
Bob:
\begin{enumerate}
    \item Chce zaszyfrować tekst jawny $P$, który dzieli na $k$ elementowych ciągów długości $n$, $P = \{P_1,\ldots,P_k\}$
    \item Szyfruje bloki tekstu jawnego otrzymując szyfrogram $C = \{c_1,\ldots,c_n\}$
    $$ \forall_{P_i \in P}, \sum_{j = 1}^n P_{ij}b_j = c_i $$
    gdzie $P_{ij}$ oznacza $j$-ty znak $i$-tego bloku tekstu jawnego
    \item Wysyła $C$ do Alicji
\end{enumerate}
Alicja:
\begin{enumerate}
    \item Po otrzymaniu szyfrogramu of Boba, Alicja oblicza $w$ - odwrotność $W$ modulo $M$ $$ wW \equiv 1 \mod M $$
    \item Używa powiązania między łatwym i trudnym problemem plecakowym $$ wb_i = a_i \mod M $$
    \item Aby odszyfrować szyfrogram $C$, obliczane jest
    \begin{align*}
        S_i = wC_i \mod M = w\sum_{j = 1}^n P_{ij}b_j \mod M
        = \sum_{j = 1}^n P_{ij}wb_j \mod M = \sum_{j = 1}^n P_{ij}a_j
    \end{align*}
    \item Ponieważ $S_i < M$ oraz $M > E$ to ostatecznie, szukanie tekstu jawnego sprowadza się do znalezienia rozwiązania $$ S_i = \sum_{j = 1}^n P_{ij}a_j $$, które można znaleźć w czasie wielomianowym ponieważ używana jest łatwa wersja problemu plecakowego
\end{enumerate}

\end{document}