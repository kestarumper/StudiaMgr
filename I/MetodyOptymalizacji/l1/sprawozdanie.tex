%%% Template originaly created by Karol Kozioł (mail@karol-koziol.net) and modified for ShareLaTeX use

\documentclass[a4paper,11pt]{article}

\usepackage[T1]{fontenc}
\usepackage[utf8]{inputenc}
\usepackage{graphicx}
\usepackage{xcolor}

\renewcommand\familydefault{\sfdefault}
\usepackage{tgheros}
\usepackage[defaultmono]{droidmono}

\usepackage{amsmath,amssymb,amsthm,textcomp}
\usepackage{enumerate}
\usepackage{multicol}
\usepackage{tikz}

\usepackage{geometry}
\geometry{left=25mm,right=25mm,%
bindingoffset=0mm, top=20mm,bottom=20mm}

\usepackage{amsmath}

\linespread{1.3}

\newcommand{\linia}{\rule{\linewidth}{0.5pt}}

% custom theorems if needed
\newtheoremstyle{mytheor}
    {1ex}{1ex}{\normalfont}{0pt}{\scshape}{.}{1ex}
    {{\thmname{#1 }}{\thmnumber{#2}}{\thmnote{ (#3)}}}

\theoremstyle{mytheor}
\newtheorem{defi}{Definition}

% my own titles
\makeatletter
\renewcommand{\maketitle}{
\begin{center}
\vspace{2ex}
{\huge \textsc{\@title}}
\vspace{1ex}
\\
\linia\\
\@author \hfill \@date
\vspace{4ex}
\end{center}
}
\makeatother
%%%

% custom footers and headers
\usepackage{fancyhdr}
\pagestyle{fancy}
\lhead{}
\chead{}
\rhead{}
\lfoot{Laboratorium 1}
\cfoot{}
\rfoot{Page \thepage}
\renewcommand{\headrulewidth}{0pt}
\renewcommand{\footrulewidth}{0pt}
%

%%%----------%%%----------%%%----------%%%----------%%%

\begin{document}

\title{Metody optymalizacji – laboratorium 1}

\author{Adrian Mucha, Politechnika Wrocławska}

\date{05/04/2014}

\maketitle

\section*{Problem 1 (Sysło, Deo, Kowalik 1993, macierz Hilberta)}

\section*{Problem 2 (Kampery)}
Rozważmy problem przemieszczenia pewnych kamperów między miastami. Kampery różnych typów $t\in T$ (rozróżniamy 'Standard' i 'Vip') należy przemieścić w zależności od zapotrzebowania $d_{c,t}$ i/lub nadmiaru $s_{c,t}$ do innych miast $c\in C$ gdzie $C$ jest zbiorem wszystkich miast, by doprowadzić do równowagi. Odpowiednio $d_{c,t}$ oznacza zapotrzebowania (zamawiający) w mieście $c$ na kampery typu $t$ oraz $s_{c,t}$ oznacza nadmiary (dostawcy). Dodatkowo kampery typu Standard można zastąpić kamperami typu Vip, ale nie na odwrót.

\subsection*{Model}
Niech $E$ oznacza zbiór możliwych połączeń między miastami $\{(c_1, c_2) \in C\times C\}$ oraz niech $l_{(c_1, c_2)\in E}$ oznacza dystans między miastem $c_1$ a $c_2$.

\subsubsection*{Zmienne decyzyjne}
Zmienne decyzyjne, wyznaczające ile należy przemieścić kamperów typu $t$ z miasta $c_1$ do miasta $c_2$ definiujemy następująco

$$
    \{\forall (c_1, c_2)\in E, \forall t\in T: x_{c_1, c_2, t} \geq 0\}
$$

\subsubsection*{Ograniczenia}
\begin{enumerate}
    \item Ilość wyjeżdżających przyczep z miasta nie może być większa niż jest w nadmiarze w danym mieście
    $$
        \forall c_1\in C, \forall t\in T: s_{c_1,t} \geq \sum_{c_2\in C} x_{c_1, c_2, t}
    $$
    \item Ilość przyjeżdżających przyczep z miasta nie może być większa niż jest zapotrzebowanie w danym mieście
    $$
        \forall c_1\in C, \forall t\in T: d_{c_1,t} \geq \sum_{c_2\in C} x_{c_2, c_1, t}
    $$
    \item Zapotrzebowanie powinno zostać wyeliminowane, a kampery typu 'Standard' można uzupełnić kamperami 'Vip'.
    $$
        \forall c_1\in C: d_{c_1, \text{Vip}} = \sum_{c_2\in C} x_{c_2,c_1,\text{Vip}} - (d_{c_1,\text{Standard}} - \sum_{c_3\in C} x_{c_3,c_1,\text{Standard}})
    $$
\end{enumerate}

\subsubsection*{Funkcja kosztu}
Cena przemieszczenia kampera jest wprost proporcjonalna do odległości między miastami $c_1$ oraz $c_2$ zdefiniowanymi w macierzy $l_{c_1,c_2}$. Kampery należy przemieścić w taki sposób, aby zminimalizować koszt ważony

$$
    \sum_{(c_1, c_2)\in E} \sum_{t\in T} w_{t} \cdot l_{c_1, c_2} \cdot x_{c_1, c_2, t}
$$

gdzie $x_{c_1, c_2, t}$ oznacza liczbę transportowanych kamperów typu $t$ z miasta $c_1$ do $c_2$ natomiast $w_t$ oznacza współczynnik kosztu za dany typ kampera. Współczynnik kosztu kampera typu Vip jest droższy o $15\%$.



\section*{Problem 3 (Przedsiębiorstwo)}
Pewna firma produkuje cztery mieszanki - produkty końcowe $P$. Dwa z nich są produktami podstawowymi $P'\subset P$ powstającymi jako mieszanka trzech surowców $S$.
Surowce te miesza się w odpowiednich proporcjach by wytworzyć produkt $P$. Każdy z tych produktów ma swoją cenę zbytu $w_s$ za kg jak i również każdy surowiec ma swoją cenę kupna $c_s$. Jednocześnie są nałożone ograniczenia na minimalną oraz maksymalną ilość surowców jaką można kupić.

Z samej natury procesu produkcji wynika fakt, że tylko pewna część każdego z surowców
użytych do produkcji produktów podstawowych $P'$ wchodzi bezpośrednio do tych produktów.
Reszta (odpady), których ilość wyraża się każdorazowo poprzez znany współczynnik strat
$o_{s, p'}$, może być albo użyta ponownie - do produkcji produktów drugorzędnych 'C' i 'D' -
albo zniszczona na koszt firmy. Kosz niszczenia surowca z produktu opisuje $d_{s,p'}$.

Drugorzędny produkt 'C' otrzymuje się poprzez wymieszanie dowolnych ilości odpadów z
surowców 1, 2, 3 otrzymanych przy wyrobie produktu A z oryginalnym surowcem 1, przy
czym ten ostatni musi stanowić (wagowo) dokladnie $20\%$ mieszanki. Podobnie, drugorzędny
produkt D otrzymuje się poprzez wymieszanie dowolnych ilości odpadów z surowców 1, 2,
3 otrzymanych przy wyrobie produktu B z oryginalnym surowcem 2, przy czym ten ostatni
musi stanowić (wagowo) dokładnie $30\%$ mieszanki. Przy produkcji produktów drugorzędnych
nie powstają żadne odpady.

Rozwiązaniem problemu są odpowiedzi na następujące pytania:
\begin{enumerate}
    \item Ile zakupić surowców (1, 2, 3)?
    \item Jaką część surowców przeznaczyć do produkcji jakiego produktu (A, B, C, D)?
    \item Jaką część odpadów z produkcji produktów A i B zniszczyć, a jaką przeznaczyć do
    produkcji produktów drugorzędnych?
\end{enumerate}

\subsection*{Model}
Zdefiniujmy zbiory. Niech $S$ oznacza zbiór dostępnych surowców, $P$ oznacza zbiór produktów powstałych z wymieszania surowców $S$, a $P'\subset P$ oznacza zbiór produktów podstawowych, które generują odpady.

Określmy parametry.
Niech $w_p: p\in P$ oznacza zysk za kg wyprodukowanego produktu $p$.
Niech $c_s: s\in S$ oznacza cenę za kg surowca $s$.
Niech $\textit{Minimum}_s$ oraz $\textit{Maximum}_s$ oznaczają minimalną (maksymalną) ilość surowca $s\in S$ jaki trzeba (można) zakupić.
Niech $o_{s, p'}$ oznacza procent produkowanego odpadu z surowca $s$ podczas wytwarzania produktu $p'$.
Niech $d_{s, p'}$ oznacza koszt zniszczenia odpadu surowca $s$ powstałego z wytworzenia produktu $p'$.
\subsubsection*{Zmienne decyzyjne}
\begin{enumerate}
    \item $k_s: s\in S$ ile kupić kg surowca $s$
    \item $x_{s,p}: s\in S, p\in P$ ile użyć surowca $s$ do wykonania produktu $p$
    \item $y_{s,p'}: s\in S, p'\in P'$ ile odpadu powstałego z surowca $s$ i produktu $p'$ zniszczyć
    \item $z_{s,p'}: s\in S, p'\in P'$ ile odpadu powstałego z surowca $s$ i produktu $p'$ przeznaczyć na tworzenie produktów drugorzędnych $p"\in P\setminus P'$
\end{enumerate}
\subsubsection*{Ograniczenia}
\begin{enumerate}
    \item Ograniczenie kupna surowców
    $$
        \forall s\in S: \textit{Minimum}_s \leq k_s \leq \textit{Maximum}_s
    $$
    \item Zużyj nie więcej surowca niż kupiłeś
    $$
        \forall s\in S: \sum_{p\in P} x_{s,p} \leq k_{s}
    $$
    \item Na produkty drugorzędne przeznacz odpadów nie więcej niż wyprodukowałeś
    $$
        \forall s\in S, \forall p'\in P': z_{s, p'} \leq x_{s,p'}o_{s,p'}
    $$
    \item Pozbądź się wszystkich powstałych odpadów za pomocą niszczenia lub przeznaczenia na produkty drugorzędne
    $$
        \forall s\in S, \forall p'\in P': x_{s,p'}o_{s,p'} = y_{s,p'} + z_{s,p'}
    $$
    \item Produkcja A wymaga tego aby mieszanka zachowywała odpowiednie proporcje
    \begin{gather*}
        x_{1, \text{'A'}} \geq 0.2 \cdot \sum_{s\in S} x_{s,\text{'A'}} \\
        x_{2, \text{'A'}} \geq 0.4 \cdot \sum_{s\in S} x_{s,\text{'A'}} \\
        x_{3, \text{'A'}} \leq 0.1 \cdot \sum_{s\in S} x_{s,\text{'A'}}
    \end{gather*}
    \item Produkcja B wymaga tego aby mieszanka zachowywała odpowiednie proporcje
    \begin{gather*}
        x_{1, \text{'B'}} \geq 0.1 \cdot \sum_{s\in S} x_{s,\text{'B'}} \\
        x_{3, \text{'B'}} \leq 0.3 \cdot \sum_{s\in S} x_{s,\text{'B'}}
    \end{gather*}
    \item Produkcja C wymaga tego aby mieszanka zachowywała odpowiednie proporcje
    $$
        x_{1, \text{'C'}} = 0.2 \cdot (x_{1, \text{'C'}} + \sum_{s\in S} z_{s,\text{'A'}})
    $$
    \item Produkcja D wymaga tego aby mieszanka zachowywała odpowiednie proporcje
    $$
        x_{2, \text{'D'}} = 0.3 \cdot (x_{2, \text{'D'}} + \sum_{s\in S} z_{s,\text{'B'}})
    $$
\end{enumerate}

\subsubsection*{Funkcja zysku}
Należy maksymalizować zysk z procesu produkcji opisanego następującą funkcją zysku

\begin{multline*}
    zysk(k, x, y, z) = \sum_{p'\in P'} \sum_{s \in S} w_{p'} x_{s, p'} (1 - o_{s, p'}) + (\sum_{s\in S} z_{s, \text{'A'}} + x_{1, \text{'C'}}) \cdot w_{\text{'C'}} + (\sum_{s\in S} z_{s, \text{'B'}} + x_{2, \text{'D'}}) \cdot w_{\text{'D'}} \\
    - (
        \sum_{s\in S} k_s c_s + \sum_{s\in S} \sum_{p'\in P'} y_{s,p'}d_{s,p'}
    )
\end{multline*}

\end{document}