%%% Template originaly created by Karol Kozioł (mail@karol-koziol.net) and modified for ShareLaTeX use

\documentclass[a4paper,11pt]{article}

\usepackage[T1]{fontenc}
\usepackage[utf8]{inputenc}
\usepackage{graphicx}
\usepackage{xcolor}

\renewcommand\familydefault{\sfdefault}
\usepackage{tgheros}
\usepackage[defaultmono]{droidmono}

\usepackage{amsmath,amssymb,amsthm,textcomp}
\usepackage{enumerate}
\usepackage{multicol}
\usepackage{tikz}

\usepackage{geometry}
\geometry{left=25mm,right=25mm,%
bindingoffset=0mm, top=20mm,bottom=20mm}


\linespread{1.3}

\newcommand{\linia}{\rule{\linewidth}{0.5pt}}

% custom theorems if needed
\newtheoremstyle{mytheor}
    {1ex}{1ex}{\normalfont}{0pt}{\scshape}{.}{1ex}
    {{\thmname{#1 }}{\thmnumber{#2}}{\thmnote{ (#3)}}}

\theoremstyle{mytheor}
\newtheorem{defi}{Definition}

% my own titles
\makeatletter
\renewcommand{\maketitle}{
\begin{center}
\vspace{2ex}
{\huge \textsc{\@title}}
\vspace{1ex}
\\
\linia\\
\@author \hfill \@date
\vspace{4ex}
\end{center}
}
\makeatother
%%%

% custom footers and headers
\usepackage{fancyhdr}
\pagestyle{fancy}
\lhead{}
\chead{}
\rhead{}
\lfoot{Laboratorium 1}
\cfoot{}
\rfoot{Page \thepage}
\renewcommand{\headrulewidth}{0pt}
\renewcommand{\footrulewidth}{0pt}
%

%%%----------%%%----------%%%----------%%%----------%%%

\begin{document}

\title{Metody optymalizacji – laboratorium 1}

\author{Adrian Mucha, Politechnika Wrocławska}

\date{05/04/2014}

\maketitle

\section*{Problem 1 (Sysło, Deo, Kowalik 1993, macierz Hilberta)}

\section*{Problem 2 (Kampery)}
Rozważmy problem przemieszczenia pewnych kamperów między miastami. Kampery różnych typów $t\in T$ (rozróżniamy 'Standard' i 'Vip') należy przemieścić w zależności od zapotrzebowania $d_{c,t}$ i/lub nadmiaru $s_{c,t}$ do innych miast $c\in C$ gdzie $C$ jest zbiorem wszystkich miast, by doprowadzić do równowagi. Odpowiednio $d_{c,t}$ oznacza zapotrzebowania (zamawiający) w mieście $c$ na kampery typu $t$ oraz $s_{c,t}$ oznacza nadmiary (dostawcy). Dodatkowo kampery typu Standard można zastąpić kamperami typu Vip, ale nie na odwrót.

\subsection*{Model}
Niech $E$ oznacza zbiór możliwych połączeń między miastami $\{(c_1, c_2) \in C\times C\}$ oraz niech $l_{(c_1, c_2)\in E}$ oznacza dystans między miastem $c_1$ a $c_2$.

\subsubsection*{Zmienne decyzyjne}
Zmienne decyzyjne, wyznaczające ile należy przemieścić kamperów typu $t$ z miasta $c_1$ do miasta $c_2$ definiujemy następująco

$$
    \{\forall (c_1, c_2)\in E, \forall t\in T: x_{c_1, c_2, t} \geq 0\}
$$

\subsubsection*{Ograniczenia}
\begin{enumerate}
    \item Ilość wyjeżdżających przyczep z miasta nie może być większa niż jest w nadmiarze w danym mieście
    $$
        \forall c_1\in C, \forall t\in T: s_{c_1,t} \geq \sum_{c_2\in C} x_{c_1, c_2, t}
    $$
    \item Ilość przyjeżdżających przyczep z miasta nie może być większa niż jest zapotrzebowanie w danym mieście
    $$
        \forall c_1\in C, \forall t\in T: d_{c_1,t} \geq \sum_{c_2\in C} x_{c_2, c_1, t}
    $$
    \item Zapotrzebowanie powinno zostać wyeliminowane, a kampery typu 'Standard' można uzupełnić kamperami 'Vip'.
    $$
        \forall c_1\in C: d_{c_1, \text{Vip}} = \sum_{c_2\in C} x_{c_2,c_1,\text{Vip}} - (d_{c_1,\text{Standard}} - \sum_{c_3\in C} x_{c_3,c_1,\text{Standard}})
    $$
\end{enumerate}

\subsubsection*{Funkcja kosztu}
Cena przemieszczenia kampera jest wprost proporcjonalna do odległości między miastami $c_1$ oraz $c_2$ zdefiniowanymi w macierzy $l_{c_1,c_2}$. Kampery należy przemieścić w taki sposób, aby zminimalizować koszt ważony

$$
    \sum_{(c_1, c_2)\in E} \sum_{t\in T} w_{t} \cdot l_{c_1, c_2} \cdot x_{c_1, c_2, t}
$$

gdzie $x_{c_1, c_2, t}$ oznacza liczbę transportowanych kamperów typu $t$ z miasta $c_1$ do $c_2$ natomiast $w_t$ oznacza współczynnik kosztu za dany typ kampera. Współczynnik kosztu kampera typu Vip jest droższy o $15\%$.



\section*{Problem 3 (Przedsiębiorstwo)}


\end{document}