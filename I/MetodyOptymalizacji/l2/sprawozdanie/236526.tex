%%% Template originaly created by Karol Kozioł (mail@karol-koziol.net) and modified for ShareLaTeX use

\documentclass[a4paper,11pt]{article}

\usepackage[T1]{fontenc}
\usepackage[utf8]{inputenc}
\usepackage{graphicx}
\usepackage{xcolor}

\renewcommand\familydefault{\sfdefault}
\usepackage{tgheros}
\usepackage[defaultmono]{droidmono}

\usepackage{amsmath,amssymb,amsthm,textcomp}
\usepackage{enumerate}
\usepackage{multicol}
\usepackage{tikz}

\usepackage{geometry}
\geometry{left=25mm,right=25mm,%
bindingoffset=0mm, top=20mm,bottom=20mm}

\usepackage{amsmath}

\linespread{1.3}

\newcommand{\linia}{\rule{\linewidth}{0.5pt}}

% custom theorems if needed
\newtheoremstyle{mytheor}
    {1ex}{1ex}{\normalfont}{0pt}{\scshape}{.}{1ex}
    {{\thmname{#1 }}{\thmnumber{#2}}{\thmnote{ (#3)}}}

\theoremstyle{mytheor}
\newtheorem{defi}{Definition}

% my own titles
\makeatletter
\renewcommand{\maketitle}{
\begin{center}
\vspace{2ex}
{\huge \textsc{\@title}}
\vspace{1ex}
\\
\linia\\
\@author \hfill \@date
\vspace{4ex}
\end{center}
}
\makeatother
%%%

% custom footers and headers
\usepackage{fancyhdr}
\pagestyle{fancy}
\lhead{}
\chead{}
\rhead{}
\lfoot{Laboratorium 2}
\cfoot{}
\rfoot{Page \thepage}
\renewcommand{\headrulewidth}{0pt}
\renewcommand{\footrulewidth}{0pt}
%

%%%----------%%%----------%%%----------%%%----------%%%

\begin{document}

\title{Metody optymalizacji – laboratorium 2}

\author{Adrian Mucha, Politechnika Wrocławska}

\date{03/05/2020}

\maketitle

\section{Chmura rozproszonych danych}

\subsection{Model}
Dane są następujące parametry:
\begin{itemize}
    \item $T_j$ - wektor zawierający czasy potrzebne na przeszukanie $j$-tego serwera.
    \item $q_{ij}$ - macierz zawierająca informację o tym, które cechy $i$ zawiera $j$-ty serwer ($1$ - obecność informacji, $0$ - brak informacji).
    \item $k$ - ilość serwerów (wnioskowana na podstawie wektora $T_j$)
    \item $n$ - ilość cech (wnioskowana na podstawie wektora $q_{ij}$)
\end{itemize}

\subsubsection{Zmienne decyzyjne}
Zmienną decyzyjną jest wektor $\mathbf{x}$ o długości $k$ odpowiadającej liczności serwerów. Wektor decyduje czy w ostatecznym przeszukiwaniu uwzględniany jest $j$-ty serwer ($x_j = 1$) czy nie ($x_j = 0$).

\subsubsection{Ograniczenia}
Dostęp do każdej cechy w conajmniej jednym wybranym serwerze
$$ \forall_{i\in[n]} (\sum_{j=1}^{k} x_j \cdot q_{ij} \geq 1) $$

\subsubsection{Funkcja kosztu}
Dążymi do minimalizowania czasów dostępu do wszystkich serwerów tak aby odczytać wszystkie cechy. Koszt opisuje następująca funkcja
$$ \textit{min}\sum_{j=1}^{k} T_j \cdot x_j $$

\subsection{Przykładowe dane}
Dla następujących danych:
\begin{itemize}
    \item $T = [1, 2, 5, 5]$
    \item $q = \left( \begin{matrix} 1 & 0 & 1 & 0 \\ 0 & 0 & 1 & 1 \\ 0 & 1 & 0 & 1 \\ 0 & 1 & 0 & 1 \\ 1 & 0 & 1 & 0 \end{matrix} \right) $
\end{itemize}

solver GLPK znalazł rozwiązanie
$$ x = [1,0,0,1] $$

\end{document}