\subsection{Model}
Dane są następujące parametry:
\begin{itemize}
    \item $T_j$ - wektor zawierający czasy potrzebne na przeszukanie $j$-tego serwera.
    \item $q_{ij}$ - macierz zawierająca informację o tym, które cechy $i$ zawiera $j$-ty serwer ($1$ - obecność informacji, $0$ - brak informacji).
    \item $k$ - ilość serwerów (wnioskowana na podstawie wektora $T_j$)
    \item $n$ - ilość cech (wnioskowana na podstawie wektora $q_{ij}$)
\end{itemize}

\subsubsection{Zmienne decyzyjne}
Zmienną decyzyjną jest wektor $\mathbf{x}$ o długości $k$ odpowiadającej liczności serwerów. Wektor decyduje czy w ostatecznym przeszukiwaniu uwzględniany jest $j$-ty serwer ($x_j = 1$) czy nie ($x_j = 0$).

\subsubsection{Ograniczenia}
Przynajmniej jeden wybrany serwer $j$ zawiera dostęp do cechy $i$-tej
$$ \forall_{i\in[n]} (\sum_{j=1}^{k} x_j \cdot q_{ij} \geq 1) $$

\subsubsection{Funkcja kosztu}
Dążymi do minimalizowania czasów dostępu do wszystkich serwerów tak aby odczytać wszystkie cechy. Koszt opisuje następująca funkcja
$$ \textit{min}\sum_{j=1}^{k} T_j \cdot x_j $$

\subsection{Przykładowe dane}
Dla następujących danych:
\begin{itemize}
    \item $T = [1, 2, 5, 5]$
    \item $q = \left( \begin{matrix} 1 & 0 & 1 & 0 \\ 0 & 0 & 1 & 1 \\ 0 & 1 & 0 & 1 \\ 0 & 1 & 0 & 1 \\ 1 & 0 & 1 & 0 \end{matrix} \right) $
\end{itemize}

solver GLPK znalazł rozwiązanie
$$ x = [1,0,0,1] $$