\section{Biblioteka podprogramów}

\subsection{Model}
Dane są następujące parametry, znajdź optymalny zestaw podprogramów $P_{ij}$ rozwiązujący program $P$ składający się z funkcji $I$ minimalizując czas przy jednoczesnym ograniczeniu pamięci:
\begin{itemize}
    \item $m$ - ilość funkcji
    \item $n$ - ilość podprogramów
    \item $M$ - górne ograniczenie pamięci programu $P$
    \item $P_{ij}$ - biblioteka $j$-tych podprogramów obliczających $i$-tą funkcję, $i\in[m], j\in[n]$
    \item $r_{ij}$ - pamięć zużywana przez podprogram $P_{ij}$
    \item $t_{ij}$ - czas wykonania podprogramu $P_{ij}$
    \item $I \subseteq \{1,\ldots,m\}$ - funkcje składające się na program $P$
\end{itemize}

\subsubsection{Zmienne decyzyjne}
Zmienną decyzyjną jest macierz $\mathbf{x}$ o wymiarach $m \times n$ przechowującą informację o wykorzystanych podprogramach w celu wykonania programu $P$.  Mianowicie $x_{ij} = 1$ oznacza wykorzystanie podprogramu $P_{ij}$ do obliczenia $i$-tej funkcji, oraz $x_{ij} = 0$ w p.p.

\subsubsection{Ograniczenia}
\begin{itemize}
    \item Uwzględniaj tylko zlecone funkcje, jeden podprogram na funkcję $$ \forall_{i\in I} \sum_{j=1}^{n} x_{ij} = 1 $$
    \item Nie wykorzystaj więcej pamięci niż $M$ podczas całej pracy programu $P$
    $$ \sum_{i=1}^{m} \sum_{j=1}^{n} x_{ij} \cdot r_{ij} \leq M $$
\end{itemize}

\subsubsection{Funkcja kosztu}
Minimalizujemy czas pracy programu $P$
$$ f(t, x) = \textit{min} \sum_{i=1}^{m} \sum_{j=1}^{n} x_{ij} \cdot t_{ij} $$

\subsection{Przykładowe dane}
Dla parametrów
\begin{itemize}
    \item $m=3$
    \item $n=4$
    \item $M=10$
    \item $I = [1, 3]$ 
\end{itemize}
wygenerowano bibliotekę funkcji $P_{ij}$ w następujący sposób
$$ r_{ij} = i \cdot j, \hspace{10pt} t_{ij} = \Biggl\lfloor \frac{100}{r_{ij}} \Biggr\rfloor $$

solver GLPK znalazł następujące rozwiązanie o koszcie (czasie) $f(T, x) = 41$
$$ x = \left( \begin{matrix} 0 & 0 & 0 & 1 \\ 0 & 0 & 0 & 0 \\ 0 & 1 & 0 & 0 \end{matrix} \right) $$