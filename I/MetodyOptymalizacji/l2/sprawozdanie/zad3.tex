\section{Gantt}

\subsection{Model}
\begin{itemize}
    \item $m$ - ilość zadań
    \item $n$ - ilość procesorów
    \item $d_{ij}$ - macierz $m\times n$ zawierajaca czasy wykonania $i$-tego zadania na $j$-tym procesorze, $i\in[m], j\in[n]$
\end{itemize}

\subsubsection{Zmienne decyzyjne}
Zdefiniowano następujące zmienne decyzyjne
\begin{itemize}
    \item macierz $\mathbf{t}$ o wymiarach $m \times n$ przechowująca czasy rozpoczęcia $i$-tego zadania na $j$-tym procesorze.
    \item $C_{\text{max}}$ - czas zakonczenia wszystkich zadań (w szczególności ostatniego zadania na trzecim procesorze)
    \item $x_{jik}$ - pomocnicza zmienna określająca kolejność wykonywanych zadań, $j\in[n],\hspace{3pt}i,k\in[m]$
\end{itemize}

\subsubsection{Ograniczenia}
\begin{enumerate}
    \item moment rozpoczecia $i$-tego zadania na $j+1$-tym procesorze może zacząć się dopiero gdy $i$-te zadanie skończy się na $j$-tej
    $$ \forall_{i\in[m], j\in[n], i < n} \hspace{5pt} t_{i, j+1} \geq t_{ij} + d_{ij} $$
    \item tylko jedno zadanie wykonywane jest w danym momencie na $j$-tym procesorze ($B$ oznacza bardzo dużą liczbę, pełniącą funkcję strażnika)
    $$ t_{ij} - t_{kj} + B \cdot x_{jik} \geq d_{kj} $$
    $$ t_{kj} - t_{ij} + B \cdot (1 - x_{jik}) \geq d_{ij} $$
    \item $C_{\text{max}}$ jest ostatnim wykonanym zadaniem na trzecim procesorze
    $$ t_{in} + d_{in} \leq C_{\text{max}} $$
\end{enumerate}

\subsubsection{Funkcja kosztu}
Minimalizujemy $C_{\text{max}}$
$$ C_{\text{max}} = C_{\pi(n)} \longrightarrow \text{min} $$
dla pewnej permutacji zadań $\pi$, gdzie $\pi(n)$ oznacza ostatnie w kolejności wykonywania zadanie.

\subsection{Przykładowe dane}
\begin{itemize}
    \item $a$ - czas trwania zadań na procesorze 1
    \item $b$ - czas trwania zadań na procesorze 2
    \item $c$ - czas trwania zadań na procesorze 3
\end{itemize}
$$
a = [3,9,9,4,6,6,7], \hspace{10pt}
b = [3,3,8,8,10,3,10], \hspace{10pt}
c = [2,8,5,4,3,1,3]
$$
$$ d = [a\hspace{1pt}b\hspace{1pt}c] $$

otrzymujemy wynik (czasy rozpoczęcia zadań), w danej kolumnie, ułożone zostały spermutowane czasy rozpoczęcia zadań. Aby otrzymać permutację etykiet, należy posortować czasy w kolumnie i odczytać kolejność zadań

$$ t = \left( \begin{matrix} 41 & 44 & 48 \\ 17 & 32 & 35 \\ 26 & 35 & 43 \\ 0 & 4 & 12 \\ 11 & 22 & 32 \\ 35 & 47 & 50 \\ 4 & 12 & 29 \end{matrix} \right) $$

gdy posortujemy pierwszą (dowolną) kolumnę rosnąco otrzymamy następującą permutację zadań
$ Z = [4, 7, 5, 2, 3, 6, 1] $ w której $C_{\text{max}} = 51$
