%%% Template originaly created by Karol Kozioł (mail@karol-koziol.net) and modified for ShareLaTeX use

\documentclass[a4paper,11pt]{article}

\usepackage[T1]{fontenc}
\usepackage[utf8]{inputenc}
\usepackage{graphicx}
\usepackage{xcolor}

\renewcommand\familydefault{\sfdefault}
\usepackage{tgheros}

\usepackage{amsmath,amssymb,amsthm,textcomp}
\usepackage{enumerate}
\usepackage{multicol}
\usepackage{tikz}

\usepackage{geometry}
\geometry{left=25mm,right=25mm,%
bindingoffset=0mm, top=20mm,bottom=20mm}

\usepackage{amsmath}

\linespread{1.3}

\newcommand{\linia}{\rule{\linewidth}{0.5pt}}

% custom theorems if needed
\newtheoremstyle{mytheor}
    {1ex}{1ex}{\normalfont}{0pt}{\scshape}{.}{1ex}
    {{\thmname{#1 }}{\thmnumber{#2}}{\thmnote{ (#3)}}}

\theoremstyle{mytheor}
\newtheorem{defi}{Definition}

% my own titles
\makeatletter
\renewcommand{\maketitle}{
\begin{center}
\vspace{2ex}
{\huge \textsc{\@title}}
\vspace{1ex}
\\
\linia\\
\@author \hfill \@date
\vspace{4ex}
\end{center}
}
\makeatother
%%%

% custom footers and headers
\usepackage{fancyhdr}
\pagestyle{fancy}
\lhead{}
\chead{}
\rhead{}
\lfoot{Laboratorium 3}
\cfoot{}
\rfoot{Page \thepage}
\renewcommand{\headrulewidth}{0pt}
\renewcommand{\footrulewidth}{0pt}
%

%%%----------%%%----------%%%----------%%%----------%%%

\begin{document}

\title{Metody optymalizacji – laboratorium 3}

\author{Adrian Mucha, Politechnika Wrocławska}

\date{31/05/2020}

\maketitle

\section*{Wstęp}
W niniejszej pracy zaimplementowano algorytm aproksymacyjny oparty na programowaniu liniowym dla zagadnienia \textit{Generalized Assignment Problem} z użyciem pakietu \texttt{JuMP}.

Zestaw danych testowych pochodzi z biblioteki \textit{OR-Library}\footnote{http://people.brunel.ac.uk/~mastjjb/jeb/orlib/gapinfo.html}, natomiast szczegóły dotyczące algorytmu znajdują się w książce \textit{Iterative Methods in Combinatorial Optimization}\footnote{http://www.contrib.andrew.cmu.edu/ ravi/book.pdf}.

\section*{Uogólnione zagadnienie przydziału}
\subsection*{Opis problemu}
Zadaniem jest aproksymacja optymalnego przydziału zbioru maszyn (potrafiących pracować równolegle) do zbioru zadań w taki sposób by minimalizować łączny czas wykonywania przy jednoczesnym uwzględnieniu ograniczeń jakie są nałożone na maszyny.

\subsection*{Model}
\begin{itemize}
    \item $M$ - zbiór maszyn
    \item $J$ - zbiór zadań
    \item $T_i$ - $i$-ta maszyna może pracować łącznie $T_i$ jednostek czasu, $i \in M$
    \item $c_{ij}$ - koszt wykonania $j$-tego zadania na $i$-tej maszynie
    \item $p_{ij}$ - czas pracy nad zadaniem $j$-tym (żużycie zasobów) przez $i$-tą maszynę
\end{itemize}

Problem sprowadza się do stworzenia grafu dwudzielnego którego wierzchołki dzielą się na zbiory $M$ oraz $J$. Początkowo tworzony jest graf pełny $G_{M,J}$, gdzie krawędź między maszyną $i\in M$ a zadaniem $j\in J$ posiada koszt $c_{ij}$.

Celem jest znalezienie podgrafu $F\subset G$, takiego że $(\forall j\in J)\; d_F(j) = 1$. Krawędzie incydentne do zadania $j$ reprezentują do której maszyny zostało ono przypisane.

\subsubsection*{Zmienne decyzyjne}
\begin{itemize}
    \item Macierz $\mathbf{X}$, której elementy $\mathbf{x_{ij}}$ mówią o tym czy $j$-te zadanie zostało przypisane do $i$-tej maszyny.
\end{itemize}

\subsubsection*{Ograniczenia}
Niech $M' \subseteq M$ - zbiór maszyn
\begin{itemize}
    \item Każde zadanie przyporządkowano jedynie raz $$ (\forall j\in J) \sum_{e=(i,j)\in \delta(j)} x_e = 1 $$
    \item Czas wykonywania zadań na maszynie nie przekracza jego czasu dostępności $$ (\forall i\in M) \sum_{e\in \delta(i)} p_e x_e \leq T_i $$
    \item zmienne decyzyjne są nieujemne $$ (\forall e\in E)\; x_e \geq 0 $$
\end{itemize}

\subsubsection*{Funkcja kosztu}
Całkowity koszt po krawędziach grafu $G$
$$ \min \sum_{e\in E} c_e x_e $$

\subsection*{Algorytm iteracyjny}
\begin{enumerate}
    \item $E(F) \longleftarrow \emptyset,\; M' \longleftarrow M $
    \item Dopóki $J \neq \emptyset$
          \begin{enumerate}
              \item Znajdź optymany punkt ekstremalny $x$ zagadnienia $LP_{ga}$ i usuń każdą zmienną decyzyjną $x_{ij}$
              \item Jeżeli istnieje $x_ij = 1$
                    \begin{enumerate}
                        \item $F \longleftarrow F \cup \{ij\}$
                        \item $J \longleftarrow J \setminus \{j\}$
                        \item $T_i \longleftarrow T_i - p_{ij}$
                    \end{enumerate}
              \item (\textbf{Relaksacja}) Jeżeli istnieje maszyna $i$ o stopniu $d(i) = 1$ lub $d(i) = 2$ posiadająca $\sum_{j\in J} x_{ij} \geq 1$, to aktualizuj $M' \longleftarrow M' \setminus \{i\}$
          \end{enumerate}
\end{enumerate}
Powyższy algorytm zapewnia, że maszyna $i$ jest używana nie więcej niż $2T_i$ jednostek czasu swojej dostępności.

\subsection*{Wyniki}
Poniższe tabele prezentują otrzymane wyniki. $T_{\max}(F)$ jest maksynalnym odchyleniem czasu przypisanego dla maszyny, oraz kolejno $T_{\max}$ jest odpowiadającemu mu ograniczeniu.

\begin{itemize}
    \item Pierwszy wpis w tabeli posiada zaburzony czas, gdyż algorytm dopiero się "rozgrzewał"
    \item Czas wykonywania zależy od rozmiaru problemu i mieści się w granicach $5$-$40$ms
    \item Algorytm potrzebował od $5$ do $9$ iteracji aby rozwiązać dany egzemplarz
    \item Maksymalne przeciążenie maszyny nigdy nie przekroczyło $2T_i$ i w zaobserwowanych wynikach średnio były przeciążone o $25\%$, z maksymalnym odchyleniem $65\%$
\end{itemize}

\begin{tabular}{ccccccc}
    Plik              & Problem         & Iteracje & $T_{\max}(F)$ & $T_{\max}$ & $ \frac{T_{\max}(F)}{T_{\max}} $ & czas           \\ \hline
    $\text{gap1.txt}$ & $\text{c515-1}$ & $6$      & $52$          & $38$       & $1.368421052631579$              & $\text{669ms}$ \\
     & $\text{c515-2}$ & $6$      & $50$          & $37$       & $1.3513513513513513$             & $\text{5ms}$   \\
     & $\text{c515-3}$ & $5$      & $54$          & $37$       & $1.4594594594594594$             & $\text{5ms}$   \\
     & $\text{c515-4}$ & $5$      & $46$          & $36$       & $1.2777777777777777$             & $\text{5ms}$   \\
     & $\text{c515-5}$ & $5$      & $56$          & $34$       & $1.6470588235294117$             & $\text{4ms}$   \\  \hline
    $\text{gap2.txt}$ & $\text{c520-1}$ & $7$      & $49$          & $42$       & $1.1666666666666667$             & $\text{8ms}$   \\
     & $\text{c520-2}$ & $5$      & $57$          & $46$       & $1.2391304347826086$             & $\text{6ms}$   \\
     & $\text{c520-3}$ & $6$      & $63$          & $48$       & $1.3125$                         & $\text{6ms}$   \\
     & $\text{c520-4}$ & $6$      & $66$          & $52$       & $1.2692307692307692$             & $\text{6ms}$   \\
     & $\text{c520-5}$ & $5$      & $59$          & $45$       & $1.3111111111111111$             & $\text{7ms}$   \\ \hline
    $\text{gap3.txt}$ & $\text{c525-1}$ & $5$      & $74$          & $64$       & $1.15625$                        & $\text{6ms}$   \\
     & $\text{c525-2}$ & $7$      & $58$          & $48$       & $1.2083333333333333$             & $\text{8ms}$   \\
     & $\text{c525-3}$ & $5$      & $67$          & $56$       & $1.1964285714285714$             & $\text{8ms}$   \\
     & $\text{c525-4}$ & $5$      & $75$          & $64$       & $1.171875$                       & $\text{9ms}$   \\
     & $\text{c525-5}$ & $6$      & $77$          & $56$       & $1.375$                          & $\text{10ms}$  \\ \hline
    $\text{gap4.txt}$ & $\text{c530-1}$ & $6$      & $88$          & $72$       & $1.2222222222222223$             & $\text{7ms}$   \\
     & $\text{c530-2}$ & $6$      & $75$          & $62$       & $1.2096774193548387$             & $\text{8ms}$   \\
     & $\text{c530-3}$ & $5$      & $95$          & $78$       & $1.2179487179487178$             & $\text{9ms}$   \\
     & $\text{c530-4}$ & $7$      & $88$          & $78$       & $1.1282051282051282$             & $\text{11ms}$  \\
     & $\text{c530-5}$ & $8$      & $78$          & $65$       & $1.2$                            & $\text{13ms}$  \\ \hline
    $\text{gap5.txt}$ & $\text{c824-1}$ & $6$      & $52$          & $35$       & $1.4857142857142858$             & $\text{12ms}$  \\
     & $\text{c824-2}$ & $5$      & $53$          & $40$       & $1.325$                          & $\text{9ms}$   \\
     & $\text{c824-3}$ & $6$      & $46$          & $38$       & $1.2105263157894737$             & $\text{35ms}$  \\
     & $\text{c824-4}$ & $6$      & $47$          & $32$       & $1.46875$                        & $\text{9ms}$   \\
     & $\text{c824-5}$ & $5$      & $46$          & $33$       & $1.393939393939394$              & $\text{7ms}$   \\ \hline
    $\text{gap6.txt}$ & $\text{c832-1}$ & $5$      & $61$          & $45$       & $1.3555555555555556$             & $\text{9ms}$   \\
     & $\text{c832-2}$ & $5$      & $61$          & $49$       & $1.2448979591836735$             & $\text{11ms}$  \\
     & $\text{c832-3}$ & $5$      & $65$          & $45$       & $1.4444444444444444$             & $\text{11ms}$  \\
     & $\text{c832-4}$ & $6$      & $62$          & $41$       & $1.5121951219512195$             & $\text{12ms}$  \\
     & $\text{c832-5}$ & $6$      & $73$          & $50$       & $1.46$                           & $\text{12ms}$  \\
\end{tabular}

\begin{tabular}{ccccccc}
    Plik               & Problem          & Iteracje & $T_{\max}(F)$ & $T_{\max}$ & $ \frac{T_{\max}(F)}{T_{\max}} $ & czas          \\ \hline
    $\text{gap7.txt}$  & $\text{c840-1}$  & $5$      & $75$          & $63$       & $1.1904761904761905$             & $\text{13ms}$ \\
     & $\text{c840-2}$  & $6$      & $68$          & $56$       & $1.2142857142857142$             & $\text{19ms}$ \\
     & $\text{c840-3}$  & $6$      & $74$          & $57$       & $1.2982456140350878$             & $\text{13ms}$ \\
     & $\text{c840-4}$  & $6$      & $75$          & $58$       & $1.293103448275862$              & $\text{18ms}$ \\
     & $\text{c840-5}$  & $6$      & $69$          & $56$       & $1.2321428571428572$             & $\text{19ms}$ \\ \hline
    $\text{gap8.txt}$  & $\text{c848-1}$  & $5$      & $62$          & $52$       & $1.1923076923076923$             & $\text{15ms}$ \\
     & $\text{c848-2}$  & $6$      & $55$          & $48$       & $1.1458333333333333$             & $\text{18ms}$ \\
     & $\text{c848-3}$  & $6$      & $59$          & $49$       & $1.2040816326530612$             & $\text{17ms}$ \\
     & $\text{c848-4}$  & $7$      & $55$          & $47$       & $1.1702127659574468$             & $\text{25ms}$ \\
     & $\text{c848-5}$  & $8$      & $69$          & $57$       & $1.2105263157894737$             & $\text{19ms}$ \\ \hline
    $\text{gap9.txt}$  & $\text{c1030-1}$ & $6$      & $50$          & $40$       & $1.25$                           & $\text{14ms}$ \\
     & $\text{c1030-2}$ & $8$      & $47$          & $38$       & $1.236842105263158$              & $\text{17ms}$ \\
     & $\text{c1030-3}$ & $5$      & $50$          & $31$       & $1.6129032258064515$             & $\text{10ms}$ \\
     & $\text{c1030-4}$ & $7$      & $52$          & $38$       & $1.368421052631579$              & $\text{16ms}$ \\
     & $\text{c1030-5}$ & $7$      & $50$          & $34$       & $1.4705882352941178$             & $\text{16ms}$ \\ \hline
    $\text{gap10.txt}$ & $\text{c1040-1}$ & $6$      & $64$          & $49$       & $1.3061224489795917$             & $\text{18ms}$ \\
    & $\text{c1040-2}$ & $5$      & $68$          & $48$       & $1.4166666666666667$             & $\text{15ms}$ \\
    & $\text{c1040-3}$ & $6$      & $60$          & $44$       & $1.3636363636363635$             & $\text{19ms}$ \\
    & $\text{c1040-4}$ & $6$      & $52$          & $44$       & $1.1818181818181819$             & $\text{18ms}$ \\
    & $\text{c1040-5}$ & $7$      & $59$          & $51$       & $1.1568627450980393$             & $\text{20ms}$ \\ \hline
    $\text{gap11.txt}$ & $\text{c1050-1}$ & $7$      & $78$          & $61$       & $1.278688524590164$              & $\text{22ms}$ \\
    & $\text{c1050-2}$ & $9$      & $82$          & $61$       & $1.3442622950819672$             & $\text{31ms}$ \\
    & $\text{c1050-3}$ & $6$      & $92$          & $74$       & $1.2432432432432432$             & $\text{23ms}$ \\
    & $\text{c1050-4}$ & $7$      & $92$          & $70$       & $1.3142857142857143$             & $\text{41ms}$ \\
    & $\text{c1050-5}$ & $7$      & $74$          & $61$       & $1.2131147540983607$             & $\text{20ms}$ \\ \hline
    $\text{gap12.txt}$ & $\text{c1060-1}$ & $7$      & $92$          & $78$       & $1.1794871794871795$             & $\text{29ms}$ \\
    & $\text{c1060-2}$ & $6$      & $85$          & $72$       & $1.1805555555555556$             & $\text{24ms}$ \\
    & $\text{c1060-3}$ & $5$      & $82$          & $66$       & $1.2424242424242424$             & $\text{21ms}$ \\
    & $\text{c1060-4}$ & $8$      & $92$          & $74$       & $1.2432432432432432$             & $\text{35ms}$ \\
    & $\text{c1060-5}$ & $6$      & $92$          & $72$       & $1.2777777777777777$             & $\text{24ms}$
\end{tabular}


\end{document}