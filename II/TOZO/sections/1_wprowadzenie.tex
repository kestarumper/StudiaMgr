\section{Wprowadzenie}
Mając ważony, nieskierowany graf $G$ i dodatnią liczbę $D$ w problemie \textit{Bounded-Diameter Minimum Spanning Tree (BDMST)} szukamy najniższego kosztem drzewa rozpinającego spośród wszystkich drzew rozpinających $G$, których ścieżki składają się z co najwyżej $D$ krawędzi. Formalnie \textit{BDST} jest drzewem $T \subset E$ na $G = (V, E)$, którego średnica jest nie większa niż $D$. \textit{BDMST} ma na celu znalezienia drzewa rozpinającego o minimalnym koszcie $w(T) = \sum_{e\in T} w(e)$. Zawężając do grafów Euklidesowych, czyli takich w których wierzchołki są punktami na przestrzeni Euklidesowej a wagi krawędzi reprezentują dystans między parami wierzchołków nazywamy \textit{Euclidean BDMST}.

Problem BDMST jest NP-trudny dla $4 \leq D < |V| - 1$, oraz trudny w aproksymacji co motywuje w poszukiwaniach efektywnej strategii opartej na heurystykach, które potrafią szybko znaleźć BDST o niskim koszcie.

Oczywistym zastosowaniem \textit{EBDMST} jest znalezienie najtańszej sieci kabli lub rur by połączyć zbiór miejsc zakładając, że koszt połączenia zależny jest od jego długości.

%Innym zastosowaniem jest rozwiązanie aproksymacyjne Euklidesowego problemu Komiwojażera (\textit{Euclidean traveling salesman problem}). Realistycznie można rozwiązać ten problem 2-aproksymacją poprzez policzenie \textit{EBDMST} i poruszanie się po brzegu obejmującym całe drzewo.

\subsection{NP-zupełność}

W książce \cite[p.~206]{10.5555/574848} Garey and Johnson pokazali, że Bounded diameter spanning tree problem jest NP-trudny.

Mając ważony graf nieskierowany $G = (V, E)$ oraz dwa parametry $D$ i $C$, decyzja \textit{czy drzewo rozpinające o koszcie $C$ oraz średnicy ograniczonej z góry przez $D$ istnieje} jest NP-zupełny \cite{DBLP:conf/compgeom/HoL89}.

BDBCST jest NP. Najcięższą ścieżką musi być ścieżką łączącą dwa liście, więc możemy zgadnąć $n-1$ krawędzi i sprawdzić czy drzewo spełnia ograniczenia w czasie wielomianowym.


