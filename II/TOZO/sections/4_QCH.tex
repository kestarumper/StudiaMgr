\section{Quadrant Centers-based (QCH)}
Zachłanność heurystyk OTTC oraz CBTC sprawia że "kręgosłup" (pień) drzewa zazwyczaj składa się z krótkich krawędzi co wymusza na niektórych aby kolejno dołączyły następny wierzchołek za pomocą krawędzi długiej co zwiększa koszt całkowity. LSoC łagodzi ten problem do pewnego stopnia szukając lepszych połączeń po zbudowaniu drzewa.

Inne podejście proponuje rozpocząć od empirycznego wybrania korzenia drzewa oraz dodania kilku wierzchołków do drzewa, które utworzą kręgosłup składający się z małej liczby wierzchołków połączonych relatywnie długimi krawędziami. Pozostałe wierzchołki dołącza się do BDST zachłannie lub za pomocą wcześniej wspomnianej heurystyki CBLSoC.

Tym podejściem kieruje się heurystyka QCH, której głównym zamysłem jest podział wierzchołków na kwadranty. W wariancie Euklidesowym, gdzie modelujemy wierzchołki jako punkty porozrzucane na kwadracie jednostkowym, możemy zastosować heurystykę CBLSoC do wyznaczenia korzenia. Podobnie, jeżeli średnica jest parzysta, to do centrum dodaje się drugi taki wierzchołek, który ma najniższy średni koszt to reszty wierzchołków i dołączony krawędzią o najniższym koszcie. Pozostałe wierzchołki grafu są grupowane w \textit{kwadranty} opisanej jako macierz $M\times M$ (elementy macierzy są kwadrantami), takiej że $2 \leq M \leq \sqrt{N}$. W każdym kwadrancie wierzchołek o najniższym średnim koszcie do reszty wierzchołków spośród tego kwadranta jest dołączany do kręgosłupa drzewa. Następnie reszta wierzchołków jest dołączana zachłannie lub za pomocą heurystyki CBLSoC zachowując warunek na średnicę.

Obie heurystyki próbują wybrać najlepszy kręgosłup zmieniając liczbę kwadrantów ($[2,n]$, gdzie $n$ to ilość wierzchołków). Heurystyka zwraca najniższe kosztem BDST otrzymane w tej procedurze. Tworzenie kręgosłupa wymaga $\mathcal{O}(n^2)$. Przebieg heurystyki po wszystkich ilościach kwadrantów trwa $\mathcal{O}(\sqrt{n})$ co daje całkowity czas $\mathcal{O}(n^2 \sqrt{n})$ \cite{DBLP:journals/informaticaSI/PatvardhanPS15}.